\documentclass[12pt]{article}

% --------------------------------------------------
% PACKAGES
% --------------------------------------------------
\usepackage[margin=1in]{geometry} % Adjust margins
\usepackage{amsmath, amssymb, amsthm} % Math typesetting
\usepackage{enumerate} % For customized lists
\usepackage{hyperref}  % Optional: clickable references/links
\usepackage{biblatex}
\bibliography{references} 
% --------------------------------------------------
% THEOREM STYLES & NEW ENVIRONMENTS
% --------------------------------------------------
\newtheorem{theorem}{Theorem}[section]
\newtheorem{lemma}{Lemma}[section]
\newtheorem{proposition}{Proposition}[section]
\newtheorem{corollary}{Corollary}[section]

\theoremstyle{definition}
\newtheorem{definition}{Definition}[section]
\newtheorem{example}{Example}[section]

\theoremstyle{remark}
\newtheorem{remark}{Remark}[section]

% Custom environment for Problems
\newenvironment{problem}[1]{
    \vspace{1em}
    \noindent
    \textbf{Problem #1.} 
}{
    \vspace{1em}
}

% Custom environment for Solutions (ends with a box)
\newenvironment{solution}{
    \noindent
    \textbf{Solution.}\\
}{
    \quad \hfill $\square$
    \vspace{1em}
}

% --------------------------------------------------
% DOCUMENT INFO
% --------------------------------------------------
\title{The Real Numbers: Cantor}

\date{\today}

% --------------------------------------------------
% BEGIN DOCUMENT
% --------------------------------------------------
\begin{document}

\maketitle

\noindent 

\bigskip

% --------------------------------------------------
% PART A
% --------------------------------------------------

\begin{problem}{(Week 2, 1.3.7)}
    Prove that if \(a\) is an upper bound for \(A\), and if \(a\) is also an element of \(A\), 
    then it must be that \(a = \sup A\).
\end{problem}

\begin{solution}
    % Write your revised solution here.
    Let $s = supA$. $a$ is an upper bound for A and by definition $1.3.2$ $(ii)$ \cite{abbott2015understanding} it follows that
    $s \leq a$. However, since s is also an upper bound, definition $1.3.1$ \cite{abbott2015understanding} implies that $x \leq s$ $\forall x \in A$. We know $a \in A$, then $a \leq s$. The two inequalities are satisfied only if $s=a$. Hence $a = supA$.

\end{solution}

\begin{problem}{(Week 3, 1.6.1)}
    Show that $(0, 1)$ is uncountable if and only if $\mathbb{R}$ is uncountable. 
\end{problem}

\begin{solution}
    % Write your revised solution here.
    First I will show that the two sets have the same cardinality - $(0, 1) \sim \mathbb{R}$. By definition $1.5.2$ \cite{abbott2015understanding} the set $(0, 1)$ has the same cardinality as $\mathbb{R}$ if there exists a mapping $f:(0, 1) \rightarrow{\mathbb{R}} $ that is bijective. By definition $1.5.1$ \cite{abbott2015understanding} the function $f:(0, 1) \rightarrow{\mathbb{R}} $ is a bijection if $(i)$ for any two elements $a_1, a_2 \in (0, 1)$ implies $f(a_1) \neq f(a_2)$ in $\mathbb{R}$ (injection) and $(ii)$ given any $b \in \mathbb{R}$, $\exists a \in (0, 1)$ s.t. $f(a)=b$ (surjection). One such function is $f(x) = \frac{2x-1}{x(x-1)}$ because we want two vertical asymptotes at $0$ and $1$ and symmetry at the middle of the interval which means root at $0.5$. % MAYBE PLOT LATER 
    
    (i) Injection: 
    Assume that $f(x_1)=f(x_2)$:
    $$\frac{2x_1-1}{x_1(x_1-1)} = \frac{2x_2-1}{x_2(x_2-1)}.$$
        This is equivalent to:
    $$x_2(2x_1-1)(x_2-1) = x_1(x_1-1)(2x_2-1)$$
    $$2x_2^2x_1-2x_1x_2 -x_2^2+x_2= 2x_1^2x_2-2x_1x_2 -x_1^2+x_1$$
    $$2x_2^2x_1-2x_1x_2 -x_2^2+x_2= 2x_1^2x_2-2x_1x_2 -x_1^2+x_1$$
    $$2x_2^2x_1-x_2^2+x_2= 2x_1^2x_2-x_1^2+x_1$$
    $$2x_2^2x_1 - 2x_1^2x_2 +x_2 - x_1 =  x_2^2-x_1^2$$
    $$2x_2x_1(x_2 - x_1) +(x_2 - x_1) - (x_2-x_1)(x_2+x_1)= 0$$
    $$(x_2 - x_1)(2x_2x_1 +1 - x_2 - x_1)= 0$$
    Here, one obvious solution is $x_2=x_1$. To prove the injection, we have to show that there are no solutions that satisfy the following for values of $x_1, x_2 \in (0, 1)$:
    $$2x_1x_2 +1 - x_1 - x_2= 0.$$
    This is equivalent to: 
    $$x_2 = \frac{x_1-1}{2x_1-1}.$$
    
    Let's look at the following cases and show that if $x_1 \in(0,1)$ it is impossible for $x_2$ to be within that interval as well.
    
    Case 1: $\frac{1}{2}<x_1<1$. Thus, $-\frac{1}{2}<x_1-1<0$. $1<2x_1<2$ and $1-1<2x_1-1<2-1$. Hence, the numerator is strictly negative and the denominator is strictly positive. This means $x_2$ must be negative and it is not between $(0, 1)$. 
    
    Case 2: $0<x_1<\frac{1}{2}$. Now let's show that $\frac{x_1-1}{2x_1-1}$ is above 1 for every $0<x_1<\frac{1}{2}$:
    $$\frac{x_1-1}{2x_1-1} > 1$$
    $$\frac{x_1-1}{2x_1-1} - \frac{2x_1-1}{2x_1-1}> 0 $$
    $$\frac{x_1-1 - (2x_1-1)}{2x_1-1}> 0 $$
    $$\frac{-x_1}{2x_1-1}> 0 $$ 
    $$\frac{x_1}{2x_1-1}< 0 $$ 
    Solving the inequality by the sign-line method, it is satisfied only for $x_1 \in (0, \frac{1}{2})$. This means that the $x_2 >1$. 
    Hence, we have shown that they are no solutions for the equation $2x_1x_2 +1 - x_1 - x_2= 0$ for $x_1, x_2 \in (0, 1)$. Thus, the only possible solution for $f(x_1) = f(x_2)$ for values of $x_1, x_2 \in (0, 1)$ is when $x_1=x_2$. This concludes the proof of $f$ being an injective function. 
        
    (ii) Surjection:
    
    To show the onto property of $f$, let $y \in \mathbb{R}$. We claim there is some $x \in (0,1)$ such that 
\[
f(x) \;=\; \frac{2x - 1}{x(x - 1)} \;=\; y.
\]
Rearranging the equation:
\[
y \;=\; \frac{2x - 1}{x(x-1)}
\quad\Longleftrightarrow\quad
y\, x(x-1) \;=\; 2x - 1.
\]
Distributing the left-hand side gives:
\[
y\, x^2 - y\, x \;=\; 2x - 1
\quad\Longleftrightarrow\quad
y\, x^2 \;-\; (y + 2)\, x \;+\; 1 \;=\; 0.
\]
Define the continuous polynomial function:
\[
G(x) \;=\; y\, x^2 - (y + 2)\,x + 1.
\]
We now check $G$ at $x=0$ and $x=1$:
\[
G(0) = 1, 
\qquad
G(1) = y \cdot 1^2 - (y+2)\cdot 1 + 1 = y - y - 2 + 1 = -1.
\]
Since $G(0) = 1 > 0$ and $G(1) = -1 < 0$ and $G$ is polynomial so it must be continuous, the Intermediate Value Theorem guarantees 
the existence of some $c \in (0,1)$ such that $G(c) = 0$. In other words,
\[
G(c) 
\;=\; 
y\, c^2 - (y+2)\,c + 1 
\;=\; 
0,
\]
so
\[
y 
\;=\; 
\frac{2c - 1}{c(c - 1)} 
\;=\; 
f(c).
\]
Hence for any $y \in \mathbb{R}$, we found $c \in (0,1)$ with $f(c) = y$, 
proving that $f$ is onto $\mathbb{R}$. We have shown that there exists a bijective mapping $f: (0, 1)\rightarrow{\mathbb{R}}$. This means that $(0, 1) \sim \mathbb{R}$. Then if $(0, 1)$ is uncountable, it follows that $\mathbb{R}$ is uncountable. I show in $1.5.5.b)$ that $(0, 1) \sim \mathbb{R}$ is equivalent to $\mathbb{R} \sim (0, 1)$ and thus, if $\mathbb{R}$ is uncountable, then $(0, 1)$ is uncountable.
    

\end{solution}



\begin{problem}{(1.5.5)}
\begin{enumerate}[(a)]
    \item Why is $A\sim A$ for every set $A$?
    \item Given sets $A$ and $B$, explain why $A\sim B$ is equivalent to asserting $B\sim A$.
     \item For three sets $A$, $B$, and $C$, show that $A\sim B$ and $B\sim C$ implies $A\sim C$. 
    
\end{enumerate}
\end{problem}

\begin{solution}
\noindent
\textbf{(a)} 
From Definition 1.5.2 \cite{abbott2015understanding}, two sets $A$ and $B$ have the same cardinality if there is a function $f: A \to B$ that is both $1$--$1$ (injective) and onto (subjective). An obvious choice is the \emph{identity} function
\[
f(x) \;=\; x \quad \text{for all } x \in A.
\]
\begin{itemize}
    \item Injective: 
    Suppose $f(x_1) = f(x_2)$. Then $x_1 = x_2$, so $f$ is injective.
    \item Surjective:
    For each $y \in A$, we can find some $x \in A$ such that $f(x) = y$.  But if we take $x = y$, then $f(x) = x = y$. Thus $f$ is onto.
\end{itemize}
Hence $f$ is a bijection and $A\sim A$. 

\medskip

\noindent
\textbf{(b)} 
To show that both statements are equivalent, we need to show both directions: if $A \sim B$, then $B \sim A$ and the opposite. Let's first address the forward direction. If $A \sim B$, then $\exists f: A\rightarrow{B}$ and f is bijective. This implies that the inverse function $f^{-1}:B\rightarrow{A}$ is well defined. To show its injectivity, we assume $f^{-1}(y_1)=f^{-1}(y_2)$ for some $y_1, y_2 \in B$. We can represent it as: $x_1 = f^{-1}(y_1)=f^{-1}(y_2) = x_2$. Applying $f$ to both sides: $f(x_1) = f(x_2)$. We have defined $x_1=f^{-1}(y_1)$ and $x_2=f^{-1}(y_2)$. Thus, $f(x_1)=y_1$ and $f(x_2)=y_2$. We have shown that if $f^{-1}(y_1)=f^{-1}(y_2)$, then $y_1=y_2$. Therefore, $f^{-1}$ is injective. Since $f$ is onto, then $\forall y \in B,\; \exists x\in A$: $f(x)=y$. Applying the inverse, we get $f^{-1}(f(x))=x, \forall x \in A.$ This means that $\forall x \in A$, we can find $y \in B$. s.t. $f^{-1}(f(x))=f^{-1}(y)=x$. Hence, $f^{-1}$ is surjective. $f^{-1}$ is both onto and 1-1, then it is bijective. It follows if $A \sim B$, then $B \sim A$. For the reverse direction, if $B \sim A$, then $A \sim B$. In broader terms we have demonstrated that if $f: A \rightarrow{B}$ is bijective, then the inverse $f^{-1}: B \rightarrow{A}$ is also bijective. Let $g:B\rightarrow{A}$. By the same argument as before, the inverse function $g^{-1}:A \rightarrow{B}$ is bijective. Thus, if $B \sim A$, then $A \sim B$. Thus, the $\sim$ relation is symmetric.

\medskip

\noindent
\textbf{(c)} 
We need to show that there is a bijective function between $A$ and $C$. Let $g: A \to B$ be a bijection and $f: B \to C$ be a bijection. Then the composition is another function: $h=g \circ f : A \to C$. To show injectivity of h, assume $h(x_1)=h(x_2)$ for some $x_1, x_2 \in A$. However, $h(x)=f(g(x)) \; \forall x \in A$. Then, $h(x_1)=f(g(x_1))=h(x_2)=f(g(x_2))$. $f:A\rightarrow{B}$ is bijective and thus, injective. This implies if $f(g(x_1))=f(g(x_2))$, then $g(x_1)=g(x_2)$. Analogically, if $g(x_1)=g(x_2)$, $x_1=x_2$. Thus, we showed directly given $h(x_1)=h(x_2)$, $x_1=x_2$. The function $h:A\rightarrow{C}$ is injective. Since $f$ and $g$ are both surjective, then $\forall y \in B, \ \exists x \in A, \; \text{s.t. } g(x)=y$ and $\forall z \in C, \ \exists y \in B, \; \text{s.t. } f(y)=z$. That is, t$\forall z \in C, \ \exists x \in A, \; \text{s.t. } f(g(x))=z$. Hence, $h$ is both surjective and injective, thus bijective. Therefore, $A$ and $C$ have the same cardinality.
\end{solution}


\begin{problem}{(1.5.6)}
    \begin{enumerate}[(a)] \item Give an example of a countable collection of disjoint
open intervals.

    \item Give an example of an uncountable collection of disjoint open intervals or argue that no such collection exists.
\end{enumerate}
\end{problem}

\begin{solution}
    \begin{enumerate}[(a)] \item We can construct a countable collection of disjoint open intervals by taking the intervals between every two consecutive natural numbers: $$(0, 1), (1, 2), ..., (n-1, n), ...$$
    $$I_n = (n-1, n) \; \forall n \in \mathbb{N}.$$
    It is obvious that these intervals are non-intersecting since each interval is separated by the integers on the number line. Now, I will demonstrate formally why this collection is countable. Define \(f: \mathbb{N} \to \mathcal{S}\) by:
    \[
    f(n) = I_n.
    \]
    \begin{itemize}
        \item Injectivity: If \(f(n) = f(m)\), then \((n-1, n) = (m-1, m)\), so \(n = m\).
        \item Surjectivity: For every \(I_k \in \mathcal{S}\), \(f(k) = I_k\).
    \end{itemize}
    Since \(f\) bijectively maps \(\mathbb{N}\) to \(\mathcal{S}\): $\mathbb{N} \sim \mathcal{S} $, and from $1.5.5$ we know this is equivalent to $\mathcal{S} \sim \mathbb{N}$, then \(\mathcal{S}\) is countable.

    \item Suppose such set $\mathcal{T}$ exists and $I_n$ is one element of $\mathcal{T}$. Theorem $1.4.3.$ \cite{abbott2015understanding} states that for every two real numbers $a$ and $b$ with $a < b$, there exists a rational number $r$ satisfying $a < r < b$. That is, $\exists r \in \mathbb{Q}$ s.t. $r \in I_n$. Since every interval $I_n$ in $\mathcal{T}$ contains at least one rational number, we can define a function:
    $$f: \mathcal{T}\rightarrow{\mathbb{Q}}, f(I_n)=r_n,$$ where $r_n \in \mathbb{Q}$. The function is injective because no two disjoint intervals can contain the same rational number. Since $\mathbb{Q}$ is countable, the image $f(\mathcal{T})$ is a subset of $\mathbb{Q}$, then according to Theorem $1.5.7$ \cite{abbott2015understanding} $f(\mathcal{T})$ is also countable. By the definition of an injective function, $\mathcal{T}$ must also be countable. This contradicts the initial supposition that $\mathcal{T}$ is uncountable. Therefore, there is no uncountable collection of disjoint open intervals in $\mathbb{R}$.

\end{enumerate}
\end{solution}



\begin{problem}{(1.6.7)}
Return to the particular functions constructed in Exercise 1.6.6 and 
construct the subset \(B\) that results using the preceding rule. 
In each case, note that \(B\) is not in the range of the function used.
\end{problem}

\begin{solution}
Looking at the functions from $1.6.6.a)$ we have the set $A = \{a, b, c\}$ and the power set $P(A)$. The two injective mappings are $f:A\rightarrow{P(A)}$ for which $f(a)=\{a\}, f(b)=\{b\}, f(c)=\{c\}$ and $g:A\rightarrow{P(A)}$ for which $g(a)=\{b, c\}, g(b)=\{a, c\}, g(c)=\{a, b\}$. Constructing the set $B=\{a \in A:a \notin f(a)\}$ for $f$ would be $B=\emptyset$, because $a \in f(a)=\{a\}$ and the same for $b$ and $c$. However, looking at the mapping $g$, $a \notin g(a)=\{b, c\}$,  $b \notin g(b)=\{a, c\}$, $c \notin g(c)=\{a, b\}$. Hence, $B=\{a, b, c\}$.
In $1.6.6.b)$ we are given the set $C=\{1, 2, 3, 4\}$ and an example of a 1-1 map $h:C\rightarrow{P(C)}$ is: $$h(1) = \{1\}, 
h(2) = \{1, 3\}, h(3) = \{4\}, h(4) = \{1, 2, 3, 4\}.$$ The set $B$ in this case is $B=\{2, 3\}$, because $2 \notin h(2)$ and $3 \notin h(3)$.
\end{solution}

\begin{problem}{(1.6.8)}
\begin{enumerate}[(a)] \item First, show that the case \(a' \in B\) leads to a contradiction.
    \item Now, finish the argument by showing that the case \(a' \notin B\) 
          is equally unacceptable.
\end{enumerate}
\end{problem}

\begin{solution}
\begin{enumerate}[(a)] \item  Since we have assumed that our function $f: A\rightarrow{P(A)}$ is onto and $B = \{a \in A: a \notin f(a)\}$, then $B \subset A$. There must be some $a' \in A, \; s.t. \; f(a')=B$. If $a' \in B$, then by definition of $B$, $a' \notin f(x)=B$. We get the contradiction that $a'$ is not an element of $B$, which is a contradiction. 


    \item Assume $a' \notin B$, then $a' \in f(x)=B$, then we get the contradiction that $a'$ must be an element of $B$ which contradicts the supposition. Therefore, $f: A \rightarrow{P(A)}$ is not surjective.
\end{enumerate}
\end{solution}


\printbibliography
% --------------------------------------------------
% END DOCUMENT
% --------------------------------------------------
\end{document}

